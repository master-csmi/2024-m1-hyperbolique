\documentclass[10pt]{beamer}
\usepackage{graphicx}
\usepackage{adjustbox}
\usepackage[backend=biber]{biblatex}
\addbibresource{references.bib}

\usetheme{Copenhagen}
\setbeamertemplate{navigation symbols}{}

\title[Adaptive Implicit Schemes for Hyperbolic Equations]{Adaptive Implicit Schemes for Hyperbolic Equations}
\author[]{Antoine Regardin -- Supervisors: Emmanuel Franck, Andrea Thomann}

\begin{document}

\frame{\titlepage}

\begin{frame}{Introduction}
    We want to study Hyperbolic PDEs such as the linear Advection Equation:
    \begin{align*}
        \partial_t u + a\partial_x u = 0
    \end{align*}
    But with a function $u$ carrying a discontinuity (= a shock) in space! (and smooth everywhere else.)
    \begin{figure}
        \centering
        \includegraphics[width=0.5\textwidth]{u0-example.png}
        %\caption{}
    \end{figure}
    Such continuity can be used to modelize a boundary between two materials.
\end{frame}

\begin{frame}
\frametitle{Theta Schemes}
Usually, representing a discontinuity in space is achievable without issues, but implementing a solver in time is another problem. %because in our case a shock in space is also a shock in time.
\\ A fixed scheme that works well for a certain step may not for the next ones...
\vspace{12pt}

A Theta Scheme for this equation can be written like this:
\begin{align*}
    \frac{u_i^{n+1} - u_i^n}{\Delta t} + a\theta\partial_t u^{n+1} + a(1-\theta)\partial_x u^n = 0
\end{align*}
%We notice that this is a hybrid between Explicit and Implicit schemes. %When $\theta = \frac{1}{2}$, we obtain a Crank-Nicholson method.

The idea is to use $\theta$ as a parameter of our model responsible of handling the shock in time, in order to have a better scheme.

\end{frame}

\begin{frame}{Self-Adaptive Theta Scheme}

    We want a small model that varies $\theta$ at each time step. We can define these bounds and conditions:
    \begin{align*}
        \theta_i^{n+1} = \begin{cases}
            max(\theta_{min}, h(u_i^n, u_i^{n+1})) & \text{if } |u_i^{n+1} - u_i^n| > \epsilon \\
            \theta^* & \text{else} 
        \end{cases}
    \end{align*}
    %And we now have a small model that varies $\theta$ at each step in order to find a better scheme.
    We can visualize this type of scheme resolution with various values of $\theta_{min}$ and $\theta^*$:
    \begin{figure}
        \centering
        \includegraphics[width=0.6\textwidth]{SAThSc-1sttests.png}
        %\caption{}
    \end{figure}
    We can already see some overlearning!

\end{frame}

\begin{frame}{Goals}
        \begin{itemize}
            \item Study the variations of the basic Theta scheme in function of $\theta$.
            \item Find an optimal $\theta$ for the initial PDE through heuristic methods.
            \item Implement a Self-Adaptive Theta Scheme. Experiment with $\theta$ variations.
            \item Study a scheme based on the flux formula in space, with the Crank-Nicholson adaptive scheme in time.
            \item Expand the method to more complex Hyperbolic systems, non-linear Transport for example.
        \end{itemize}
\end{frame}

\begin{frame}
\frametitle{Bibliography}
\nocite{*}
\printbibliography
\end{frame}

\end{document}
