\documentclass[12pt]{article}
\usepackage{amsmath}
\usepackage{amsfonts}
\usepackage{graphicx,psfrag,epsf}
\usepackage{enumerate}
\usepackage{float}

\usepackage[utf8]{inputenc}
\usepackage[style=authoryear,backend=biber]{biblatex}
\addbibresource{references.bib}
\DeclareSourcemap{
  \maps[datatype=bibtex]{
    \map{
      \step[fieldset=annotation, fieldsource=annotation]
      \step[fieldset=note, origfieldval]
    }
  }
}

\usepackage{listings}
\usepackage{xcolor}
\lstset{ % General setup for the package
    language=Python,
    basicstyle=\small\sffamily,
    numbers=left,
    numberstyle=\tiny,
    frame=single,
    tabsize=4,
    columns=fixed,
    showtabs=false,
    keepspaces,
    keywordstyle=\color{blue}\bfseries,
    commentstyle=\color{gray},
    stringstyle=\color{orange},
    showstringspaces=false,
    breaklines=true,
    tabsize=4,
    escapeinside={*@}{@*},
    xleftmargin=14mm,
    xrightmargin=14mm,
}

\addtolength{\oddsidemargin}{-.75in}%
\addtolength{\evensidemargin}{-.75in}%
\addtolength{\textwidth}{1.7in}%
\addtolength{\textheight}{1.5in}%
\addtolength{\topmargin}{-.8in}%


\begin{document}
\def\spacingset#1{\renewcommand{\baselinestretch}%
{#1}\small\normalsize} \spacingset{1}

\title{\bf M1 CSMI Project\\ Adaptive Implicit Schemes for Hyperbolic Equations}
\author{Antoine REGARDIN\hspace{.2cm}\\
    Supervisors: Emmanuel FRANCK, Andrea THOMANN\\
    University of Strasbourg\\ }
\date{Spring 2024}
\maketitle

\tableofcontents

\newpage

\section{Introduction: Context}
In the numerical analysis of hyperbolic problems, explicit schemes can present very bad results, and solutions obtained with implicit schemes
are too prone to dissipation. This is why Theta Schemes (including the Crank-Nicholson scheme) are an interesting compromise for their study.\\
Another issue comes to us when studying more specific problems, with discontinuous initial solutions $u_{t=0}(x)$: when using a normal theta scheme (with any $\theta$ between 0 and 1),
the numerical solution does not behave well around such discontinuities. A possible solution to this could be to make $\theta$ vary in function of each space and time cells of our mesh.\\
Indeed, representing a discontinuity in space is achievable without issues, but implementing a solver in time is another problem.
Here, as our discontinuity is moving in function of the time, "a shock in space is also a shock in time".\\
-> bad CFL: =>bad numerical solutions, but often necessary for studies
-> Riemann problems, discontinuities


-> put a figure to illustrate a Riemann problem
-> and another for another equation (Burgers, RIPA, ...)


It is important to find a solution to this issue, as local discontinuities -or singularities- can be important in modelling in fluids, materials or waveforms Physics.
For example, many phenomena can be studied at the boundary between two different physical media,
with corresponding functions presenting a jump in space.\\
To sum up, our main goal is to study an outline of numerical resolution of Transport Equations problems
involving this kind of discontinuous functions. In this project, we will work on implementing a Self-Adaptive Theta Scheme as formulated by Pr. Arbogast, 
in the context of the linear advection equation, that modelizes the linear displacement of a quantity in a simple moving environment:

\begin{align*}
    \partial_t u + a\partial_x u = 0 \text{\space \space ($a$ constant)} \tag{1}
\end{align*}
\vspace{10pt}



\section{Material and Methods}
First of all, we will only study the case of one-dimensional $\mathbb{R+} \times \mathbb{R} \rightarrow \mathbb{R}$ functions, and thus stay on a 1D spatial domain, parted with a constant time step.
All of the code will be written in Python, with the help of libraries such as $numpy$, $pyplot$, or $scipy$.

\subsection{Code}
All the useful functions and classes are stored in the file $SATh\_utilities.py$.
We use 5 classes to represent all the objects needed:
\begin{itemize}
    \item the main class $Problem$ that calls the objects of the other classes.
    \item the class $Mesh$ that contains the methods to create the discretized domain.
    \item the class $Matrices$ that creates the useful matrices for the numerical resolution. The function \emph{Iter\_Mat}
builds the iteration matrix for either the simple Theta Scheme (in this case, it is only called once) or the Self-Adaptive Theta Scheme (in this case, it is called at each time step).
This class also inherits from $Mesh$.
    \item the class $Functions$ that creates the useful functions: the item \emph{init\_sol} contains the initial function (at $t=0$) and the item \emph{exact\_sol} contains the exact solution (at $t=t_{final}$)
    \item the class $SATh\_Solver$ contains all the methods needed for the Self-Adaptive Theta method. It is called by the main class.
\end{itemize}

\subsection{Hyperbolic Equations: The Advection Equation}
Here is the formulation of the advection equation:
\begin{align*}
    \partial_t u + a\partial_x u = 0 \text{\space \space ($a$ constant)} \tag{1}
\end{align*}
The parameter $a$ characterizes the speed and direction of the flux modelized by this equation.
We will work with $a > 0$, meaning that, with the time going on, the flux will go from left to right in our spatial domain.
With $a < 0$ the flux would go from right to left, this would have no consequences on the properties of our numerical solving, but it can simplify the implementation.

We will work with Neumann boundary conditions: the values of the derivatives in time of the solution must be equal to 0:
\begin{align*}
    \partial_t u (a) = 0 = \partial_t u (b)
\end{align*}
Indeed, with this choice we will not be bothered by the behaviour of our solutions on the boundary.
This will allow us to focus on the modelling of the displacement of a perturbation strictly inside the domain. See \textit{fig. 1} for an example of the functions we want to study.
We can see that the interest is the perturbation, and that the behaviour on the boundaries is trivial.\\
We can note that with some other types of functions, we can also work with periodical conditions.

\vspace{10pt}

\subsection{The Theta Scheme for the advection equation}

For now, we will work on a 1D spatial domain of size $N$ in the interval $[a,b]$:\\
Let $u$ be a general function defining a field. We split the field u with values $ x_i \text{,} \hspace{8pt} i \in \{0, 1, 2, \ldots, N\}$
And we will write $u(t,x_i) = u_i^t$ \\
When not explicitly written, the values of $a$ and $b$ are respectively 0 and 1.\\

First, let's define the following operators for the discretisation of partial differentiation 
(note that these are dependant of the direction of the flux $a$, here we have as stated previously $a > 0$):
\begin{align*}
    & d_t u = \frac{u^{n+1} - u^n}{\Delta t} \\
    & d_x u = 
    \begin{cases} 
        \frac{u_{i}^n - u_{i-1}^n}{\Delta x} & \text{for } i \in \{1, 2, \ldots, N\} \\
        0 & \text{if } i = 0 
    \end{cases}
\end{align*}
Note that we have introduced a ghost cell at $i=0$, i.e. at the left of the domain, in order to implement the Neumann condition.
Indeed, we will thus always have $u_0^n = u_0^{n+1}$ and $\partial_t u_0 = 0$ for every $n$.\\
At the right boundary, we do not need to do the same thing, because we have restrained to the case where the flux goes from left to right, so the value in the last cell does not depend of another value at its right.

\vspace{8pt}
We can write the Theta Scheme for the equation $(1)$:
\begin{align*}
    d_t u + a\theta d_x u^{n+1} + a(1-\theta) d_x u^n = 0 \tag{2a}
\end{align*}


This scheme has a rate of convergence of order 2 in time (except for the strictly explicit case $\theta=0$ where it is 1), and order 1 in space.\\
We can see that the parameter $\theta$ defines a balance between implicit and explicit resolution.\\
The notation $(2a)$ represents a system of equations, that we now want to write as matrices: \\

\[
\begin{array}{c}
\displaystyle \frac{u^{n+1} - u^n}{\Delta t} + a\theta d_x u^{n+1} + a(1-\theta) d_x u^n = 0 \\
\Leftrightarrow \\
\displaystyle (Id + \Delta t . a \theta d_x)u^{n+1} = u^n - \Delta t .a(1-\theta)d_x u^n \\
\Leftrightarrow \\

\underbrace{
    \begin{pmatrix}
    & 1 & & (0) &\\
    & -a\theta\Delta t & 1+\theta a \Delta t & & \\
    & & \ddots & \ddots \\
    & (0) & -a\theta\Delta t & 1+\theta a \Delta t \\
    \end{pmatrix}
}_{\text{$A$}}

\underbrace{
    \begin{pmatrix}
    u_0^{n+1}\\
    u_1^{n+1}\\
    \vdots \\
    u_N^{n+1}\\
    \end{pmatrix}
}_{\text{$u^{n+1}$}}
=
\underbrace{
    \begin{pmatrix}
    u_0^n\\
    u_1^n - \Delta t .a(1-\theta)d_x u_1^n\\
    \vdots \\
    u_{N}^n - \Delta t .a(1-\theta)d_x u_{N}^n\\
    \end{pmatrix}
}_{b^n} \tag{2b}
\end{array}
\]
We remember that, for exemple, $d_x u_1 = \frac{u_1^n - u_0^n}{\Delta x}$.\\
$A$ is a square matrix depending of the values $\Delta t$, $a$ and $\theta$ which size is equal to the number of points in the mesh,
and $b^n$ a vector of the same size depending of $\Delta t$, $a$, $\theta$ and the values of $u$ at the previous time step $n$.\\
We use the \textit{csr} format of the $scipy.sparse$ library to store $A$, and we solve this system at each time iteration using the $GMRES$ algorithm function from the same library. 


\subsection{Self-Adaptive Theta Scheme for the advection equation}
We now want to use $\theta$ as a parameter to vary at each time step, in order to provide a kind of adaptation to the displacement of the discontinuous jump.
This is the principle of the Self-Adaptive Theta method. For a time step $n$ and a space cell of index $i$, we will solve:
\begin{align*}
    d_t u + a\theta_i^n d_x u^{n+1} + a(1-\theta_i^n) d_x u^n = 0 \tag{3a}
\end{align*}
In this case, the $\theta$ value can be different for two different space coordinates, or for the same space coordinate but at different time step.\\
Our numerical system changes a little bit. Now we have to take care of the variation of the different $\theta$ values:

\[
\begin{array}{c}
\underbrace{
    \begin{pmatrix}
    & 1 & & (0) &\\
    & -a\theta_1^n \Delta t & 1+\theta_1^n a \Delta t & & \\
    & & \ddots & \ddots \\
    & (0) & -a\theta_N^n\Delta t & 1+\theta_N^n a \Delta t \\
    \end{pmatrix}
}_{\text{$A^n$}}

\underbrace{
    \begin{pmatrix}
    u_0^{n+1}\\
    u_1^{n+1}\\
    \vdots \\
    u_N^{n+1}\\
    \end{pmatrix}
}_{\text{$u^{n+1}$}}
=
\underbrace{
    \begin{pmatrix}
    u_0^n\\
    u_1^n - \Delta t .a(1-\theta_1^n)d_x u_1^n\\
    \vdots \\
    u_{N}^n - \Delta t .a(1-\theta_N^n)d_x u_{N}^n\\
    \end{pmatrix}
}_{b^n} \tag{3b}
\end{array}
\]
We need to update the matrix $A^n$ at each time step, contrary to the standard Theta method.\\
The main challenge of this new method is to find an optimal value for each $\theta$. Pr.Arbogast proposes this choice function:

\begin{align*}
    \theta_i^{n+1} = \begin{cases}
        min(max(\theta_{min}, \frac{\tilde{u}_i^{n+1} - u_i^n}{u_i^{n+1} - u_i^n} ), 1) & \text{if } |u_i^{n+1} - u_i^n| > \epsilon \tag{4} \\
        \theta^* & \text{else} 
    \end{cases}
\end{align*}

We introduce $\theta_{min}$ and $\theta^*$ as new parameters of the numerical method.\\
The value $\tilde{u}_i^{n+1}$ is an interpolation of the numerical solution between two time steps.\\

\begin{figure}[H]
    \centering
    \includegraphics[width=0.7\textwidth]{indexmapu.png}
    \caption{Representation of the cells and interpolations}
\end{figure}
In the reference paper, it is suggested to use a \textit{Discontinuity Aware Quadrature} to interpolate, in order to obtain a $\tilde{u}$
that loses the less information possible about the perturbation, and thus get the $\theta$ value the most adapted to the discontinuity.
For reasons of time and complexity, we chose to implement a method that does not require to compute this interpolation (as will be seen later).\\

Now, we notice that, in $(4)$, we need the value $u_i^{n+1}$ in order to compute $\theta_i^{n+1}$,
but in the other way we need $\theta_i^{n+1}$ to compute $u_i^{n+1}$.
Mrs. Thomann suggested to use the following fixed-point iteration method to update the $\theta$ values at each time step:\\
We note $\lambda = \frac{\Delta t}{\Delta x}$, $w_i^{n+1} = u_i^{n+1} - u_i^{n}$ and $v_i^{n+1} = \tilde{u}_i^{n+1} - u_i^{n+1}$.\\
On Pr. Arbogast's paper, a formula is given in order to compute $w_i^{n+1}$ and $v_i^{n+1}$:

\begin{align*}
    w_i^{n+1} + \lambda \theta_i^{n+1} w_i^{n+1} = -\lambda(u_i^n - \theta_{i-1}^{n+1} w_{i-1}^{n+1} - u_{i-1}^{n+1}) \tag{5a}\\
    v_i^{n+1} + \frac{\lambda}{2} (\theta_i^{n+1})^2 w_i^{n+1} = -\frac{\lambda}{2} (u_i^n - (\theta_{i-1}^{n+1})^2 w_{i-1}^{n+1} - u_{i-1}^{n+1}) \tag{5b}
\end{align*}

\begin{align*}
    w_i^{(k)} + \lambda \theta_i^{(k-1)} w_i^{(k)} = -\lambda(u_i^n - \theta_{i-1}^{(k)} w_{i-1}^{(k)} - u_{i-1}^{n+1}) \tag{6a}\\
    v_i^{(k)} + \frac{\lambda}{2} (\theta_i^{(k-1)})^2 w_i^{(k)} = -\frac{\lambda}{2} (u_i^n - (\theta_{i-1}^{(k)})^2 w_{i-1}^{(k)} - u_{i-1}^{n+1}) \tag{6b}
\end{align*}


test: \textcite{arbogast2021}

\printbibliography

\end{document}
