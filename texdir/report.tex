\documentclass[12pt]{article}
\usepackage{amsmath}
\usepackage{amsfonts}
\usepackage{graphicx,psfrag,epsf}
\usepackage{enumerate}
\usepackage{natbib}
\usepackage{float}

\usepackage{listings}
\usepackage{xcolor}
\lstset{ % General setup for the package
    language=C++,
    basicstyle=\small\sffamily,
    numbers=left,
    numberstyle=\tiny,
    frame=single,
    tabsize=4,
    columns=fixed,
    showtabs=false,
    keepspaces,
    keywordstyle=\color{blue}\bfseries,
    commentstyle=\color{gray},
    stringstyle=\color{orange},
    showstringspaces=false,
    breaklines=true,
    tabsize=4,
    escapeinside={*@}{@*},
    xleftmargin=14mm,
    xrightmargin=14mm,
}

\addtolength{\oddsidemargin}{-.75in}%
\addtolength{\evensidemargin}{-.75in}%
\addtolength{\textwidth}{1.7in}%
\addtolength{\textheight}{1.5in}%
\addtolength{\topmargin}{-.8in}%


\begin{document}
\def\spacingset#1{\renewcommand{\baselinestretch}%
{#1}\small\normalsize} \spacingset{1}

\title{\bf M1 CSMI Project\\ Adaptive Implicit Schemes for Hyperbolic Equations}
\author{Antoine REGARDIN\hspace{.2cm}\\
    Supervisors: Emmanuel FRANCK, Andrea THOMANN\\
    University of Strasbourg\\ }
\date{Spring 2024}
\maketitle

\section{Introduction: Context}
In the numerical analysis of hyperbolic problems, explicit schemes present very bad results, and solutions obtained with implicit schemes
are too prone to dissipation. This is why Theta Schemes (including the Crank-Nicholson scheme) are an interesting compromise for their study.\\
Another issue comes to us when studying more specific problems, with discontinuous initial solutions $u_{t=0}(x)$: when using a normal theta scheme (with any $\theta$ between 0 and 1),
the numerical solution does not behave well around such discontinuities. A possible solution to this could be to make $\theta$ vary in function of each space and time cells of our mesh.\\
Indeed, representing a discontinuity in space is achievable without issues, but implementing a solver in time is another problem.
Here, as our discontinuity is moving in function of the time, "a shock in space is also a shock in time".\\
\vspace{10pt}
\begin{figure}[H]
    \centering
    \includegraphics[width=0.3\textwidth]{u0-example.png}
    \caption{An example of an initial function presenting a discontinuous jump.}
\end{figure}

It is important to find a solution to this issue, as local discontinuities -or singularities- can be important in modelling in fluids, materials or waveforms Physics.
For example, many phenomena can be studied at the boundary between two different physical media,
with corresponding functions presenting a jump in space.\\
To sum up, our main goal is to study an outline of numerical resolution of Transport Equations problems
involving this kind of discontinuous functions. We will particularly work on the Self-Adaptive Theta Schemes, 
first with the linear advection equation, then, if possible, with more complex problems.
\vspace{10pt}



\section{Material and Methods}
First of all, we will only study the case of one-dimensional $\mathbb{R} \times \mathbb{R} \rightarrow \mathbb{R}$ functions, and thus stay on a 1D spatial domain.
All of the code will be written in Python, with the help of libraries such as $numpy$, $pyplot$, or $scipy$.

\subsection{Hyperbolic Equations: The Advection Equation}
Here is the formulation of the advection equation:
\begin{align*}
    \partial_t u + a\partial_x u = 0 \text{\space \space ($a$ constant)} \tag{1}
\end{align*}
\vspace{10pt}
We will explore our numerical methods in the case of this PDE, because it is simpler. We will try to expand this study to more complex ones if possible.


The main idea is to start from a Theta Scheme (here for the equation (1)):
\begin{align*}
    \frac{u_i^{n+1} - u_i^n}{\Delta t} + a\theta d_t u^{n+1} + a(1-\theta) d_x u^n = 0 \tag{2}
\end{align*}
This scheme has a rate of convergence of order 2 in time (except for the case $\theta=0$ where it is 1), and order 1 in space.\\
We now want to use $\theta$ as a parameter to vary at each time step, in order to provide a kind of adaptation to the displacement of the discontinuous jump.
For example, we can define our varying $\theta$ at a time step $n+1$ as:

\begin{align*}
    \theta_i^{n+1} = \begin{cases}
        max(\theta_{min}, h(u_i^n, u_i^{n+1})) & \text{if } |u_i^{n+1} - u_i^n| > \epsilon \\
        \theta^* & \text{else} 
    \end{cases}
\end{align*}

With, for example, the reference value for $\theta$ that can follow the Crank-Nicholson method: $\theta^* = \frac{1}{2}$

\subsection{The Theta Scheme for the advection equation}
For now, we will work on a periodical spatial 1D domain of size $N$ in the interval $[a,b]$:
\begin{align*}
    \text{We split the field u with values} u_i i \in \{1, 2, \ldots, N\}
\end{align*}
When not explicitly written, the values of $a$ and $b$ are respectively 0 and 1.

First, this method defines the following operators for the discretisation of partial differentiation:
%\begin{align*}
%    \text{Let u be a general function defining a field. u^n is the state of the field at the time n.}\\
%    d\_t(u\^n) &= \frac{u\^\{n+1\} - u\^n}{\Delta t} \\
%    d\_x(u\^n) &= \frac{u\_\{i+1\}\^\{n\} - u\_\{i\}\^\{n\}}{\Delta x} \quad | i \in \{ 1, 2, \ldots, N \}
%\end{align*}


%\subsection{Self-Adaptive Theta Scheme}

%\section 




\end{document}