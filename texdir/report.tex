\documentclass[12pt]{article}
\usepackage{amsmath}
\usepackage{graphicx,psfrag,epsf}
\usepackage{enumerate}
\usepackage{natbib}
\usepackage{float}

\usepackage{listings}
\usepackage{xcolor}
\lstset{ % General setup for the package
    language=C++,
    basicstyle=\small\sffamily,
    numbers=left,
    numberstyle=\tiny,
    frame=single,
    tabsize=4,
    columns=fixed,
    showtabs=false,
    keepspaces,
    keywordstyle=\color{blue}\bfseries,
    commentstyle=\color{gray},
    stringstyle=\color{orange},
    showstringspaces=false,
    breaklines=true,
    tabsize=4,
    escapeinside={*@}{@*},
    xleftmargin=14mm,
    xrightmargin=14mm,
}

\addtolength{\oddsidemargin}{-.75in}%
\addtolength{\evensidemargin}{-.75in}%
\addtolength{\textwidth}{1.7in}%
\addtolength{\textheight}{1.5in}%
\addtolength{\topmargin}{-.8in}%


\begin{document}
\def\spacingset#1{\renewcommand{\baselinestretch}%
{#1}\small\normalsize} \spacingset{1}

\title{\bf M1 CSMI Project\\ Adaptive Implicit Schemes for Hyperbolic Equations}
\author{Antoine REGARDIN\hspace{.2cm}\\
    Supervisors: Emmanuel FRANCK, Andrea THOMANN\\
    University of Strasbourg\\ }
\date{Spring 2024}
\maketitle

\section{Context}
Local discontinuities -or singularities- can be important in modelling in fluids, materials or waveforms Physics.
For example, many phenomena can be studied at the boundary between two different physical media,
with corresponding functions presenting a jump in space.\\
In this project, our main goal is to study an outline of numerical resolution of Transport Equations problems
involving this kind of discontinuous functions. We will particularly work on the Self-Adaptive Theta Schemes, 
first with the linear advection equation, then, if possible, with more complex problems.
We will stay on a 1D spatial domain, and all of the code will be written in Python, with the help of libraries such as $numpy$, $pyplot$, or $scipy$.
\vspace{10pt}

\section{Introduction}

Here is the formulation of the advection equation:
\begin{align*}
    \partial_t u + a\partial_x u = 0 \text{\space \space ($a$ constant)} \tag{1}
\end{align*}

\vspace{10pt}
We want to study its solutions presenting a jump in space coordinates:
\begin{figure}[H]
    \centering
    \includegraphics[width=0.3\textwidth]{u0-example.png}
    \caption{An example of initial solution for our schemes.}
\end{figure}

Usually, representing a discontinuity in space is achievable without issues, but implementing a solver in time is another problem.
Indeed, here as our discontinuity is moving in function of the time, "a shock in space is also a shock in time".\\
\vspace{10pt}

The main idea is to start from a Theta Scheme (here for the equation (1)):
\begin{align*}
    \frac{u_i^{n+1} - u_i^n}{\Delta t} + a\theta d_t u^{n+1} + a(1-\theta) d_x u^n = 0 \tag{2}
\end{align*}
And use $\theta$ as a parameter to vary at each time step, in order to provide a kind of adaptation to the displacement of the discontinuous jump.
For example, we can define our varying $\theta$ at a time step $n+1$ as:

\begin{align*}
    \theta_i^{n+1} = \begin{cases}
        max(\theta_{min}, h(u_i^n, u_i^{n+1})) & \text{if } |u_i^{n+1} - u_i^n| > \epsilon \\
        \theta^* & \text{else} 
    \end{cases}
\end{align*}

With, for example, the reference value for $\theta$ that can follow the Crank-Nicholson method: $\theta^* = \frac{1}{2}$




%\section{Self-Adaptive Theta Scheme}




\end{document}