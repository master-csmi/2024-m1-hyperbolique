
% !TEX TS-program = pdflatex
% !TEX encoding = UTF-8 Unicodeo
%\documentclass[aip, pof, preprint, floatfix]{revtex4-1}
\documentclass[3p]{scrartcl}%{article}%


%PACKAGES
%\usepackage[utf8x]{inputenc}
%\usepackage{titlesec}
\usepackage{titling}
\usepackage[hmarginratio=1:1,top=32mm,left=15mm,columnsep=1cm]{geometry} %
%\usepackage{lmodern}
\usepackage{array}
\usepackage{color}
\usepackage{tabularx}
\usepackage{graphicx}
\usepackage[svgnames,table,xcdraw]{xcolor}
\usepackage{amsmath}
\usepackage{amsthm}
\usepackage{amssymb}
\usepackage{amsfonts}
\usepackage{comment}
\usepackage{appendix}
%\usepackage{moreverb}
%\usepackage{dsfont}
%\usepackage{tipa}
%\usepackage{upgreek}
%\usepackage{grffile}
\usepackage{bm}
%\usepackage{xfrac}
\usepackage{multirow}
%\usepackage{soul}
%\usepackage{ textcomp }
\usepackage{relsize} % e.g. used for \mathsmaller
\usepackage{fourier} % for the norm comand \Vert (it has smaller space between the vertical bars
%then just in ||)
\usepackage{epstopdf}
\usepackage[font=small,labelfont=bf]{caption}	% set font of caption text
\usepackage{multicol,booktabs}
% \usepackage{showlabels}
\usepackage{nomencl}	% create nomenclature
\usepackage{pbox} % figure in matrix cells
\usepackage{subcaption} %subtabs and subfigures
\makenomenclature


%% BIBLATEX options:
%%\usepackage[sorting=none]{biblatex}
%\usepackage[giveninits=true,url=false,doi=true,eprint=false,isbn=false,
%backref,backrefstyle=none,sorting=none,maxbibnames=99]{biblatex}
%\DefineBibliographyStrings{english}{%
    %	backrefpage = {Cited on p\adddot},%
    %	backrefpages = {Cited on pp\adddot}%
    %}
%
%\bibliography{library}
%% \bibliography{/home/utilisateur/Dropbox/library}
%%\bibliography{/home_pers/peshkov/Dropbox/library}
%
%\renewcommand*{\bibfont}{\footnotesize}
%
%% in order to suppress 'In:'
%\renewbibmacro{in:}{%
    %	\ifboolexpr{%
        %		test {\ifentrytype{article}}%
        %	}{}{\printtext{\bibstring{in}\intitlepunct}}%
    %}
% END BIBLATEX options.

\graphicspath{{../}}	% graphics file path


% NEW COMMANDS
\renewcommand{\tabularxcolumn}[1]{m{#1}}
\newcommand\BibTeX{{\rmfamily B\kern-.05em \textsc{i\kern-.025em b}\kern-.08em
        T\kern-.1667em\lower.7ex\hbox{E}\kern-.125emX}}


%\newcommand{\familydefault}{\sfdefault} %Change the typeface of the font
\linespread{1.05} % Line spacing - Palatino needs more space between lines




%\renewcommand{\tabularxcolumn}[1]{m{#1}}

% To have colored cited papers, hyperlinked to the
% bibiography, help to know if papers are not cited
% but in the bibliography still
\usepackage{hyperref}
\hypersetup{
    colorlinks=true,
    linkcolor=Blue, % Couleur des liens internes
    citecolor=DarkRed, % Couleur des num�ros de la biblio dans le corps
    urlcolor=blue  } % Couleur des url
%\usepackage[hyperpageref]{backref}
%usepackage[square,numbers]{natbib}
%\RequirePackage[hyperpageref]{backref}
%\backreffrench
%\renewcommand*{\backref}[1]{}  % Disable standard
%\renewcommand*{\backrefalt}[4]{% Detailed backref
    %\ifcase #1 %
    %\relax%(Not cited.)%
    %\or
    %% (Cit\'e page~#2.)%
    %(Cited page~#2.)%
    %\else
    %%(Cit\'e pages~#2.)
    %(Cited page~#2.)%
    %\fi}

%\setlength{\oddsidemargin}{.5cm} \setlength{\evensidemargin}{.5cm}
%\setlength{\textwidth}{15cm} \setlength{\textheight}{21.0cm}
%\setlength{\topmargin}{0in}


%\usepackage{fontspec}
% Specify different font for section headings
%\newfontfamily\headingfont[]{\bfseries}
%\titleformat*{\section}{\LARGE\sffamily}
%\titleformat*{\subsection}{\Large\sffamily}
%\titleformat*{\subsubsection}{\large\sffamily}
\renewcommand{\maketitlehooka}{\sffamily}


%%%%%%%% NEW COMMANDS %%%%%%%%
\newcommand{\pd}{\mathcal{\partial}}
\newcommand{\ce}{{\varepsilon}}
\newcommand{\xx}{{\boldsymbol{x}}}
\newcommand{\yy}{{\boldsymbol{y}}}
\newcommand{\zz}{{\boldsymbol{z}}}
\newcommand{\q}{\boldsymbol{q}}
\newcommand{\x}{\mbf{x}}
\newcommand{\y}{\mbf{y}}
\newcommand{\Th}{\mathcal{T}_h}				%mesh notation
\newcommand{\nij}{\mathbf{n}_{ij}}				%outward unit normal to S_ij
\newcommand{\Eel}{\mathcal{E}_{el} }			%indices set of the real cells
\newcommand{\Ebd}{\mathcal{E}_{bd} }			%indices set of the virtual cells
\newcommand{\tEel}{\widetilde{\mathcal{E}_{el}}}	%indices set of the real and virtual cells
\newcommand{\Epc}{\mathcal{E}_{pc} }			%indices set of problematic cells
\newcommand{\Unui}{\underline{\nu}(i)}		%indices set of cells linked to K_i by a side
\newcommand{\Onui}{\overline{\nu}(i)}			%indices set of every cells linked to K_i
\newcommand{\urec}{\widetilde u}				%polynomial rec sur K
\newcommand{\Urec}{\widetilde U}				%polynomial rec sur K
\newcommand{\R}{\varrho_{12}^{-1}} %NEEDED FOR POLY REC ..
\newcommand{\CPD}{\textsf{CellPD}}
%\newcommand{\un}{\textrm{1}}
\newcommand{\un}{1 \hskip -3pt \textrm{I}}
\newcommand{\Deg}[1]{ \mathsf{d}_{#1} }
\newcommand{\EPD}{\textsf{EdgePD}}
\newcommand{\FPD}{\textsf{FacePD}}
\newcommand{\EPDMeth}[1]{$\mathsf{EPD}_{\mathsf{#1}}$}
\newcommand{\mc}[1]{\mathcal{#1}}			% Simplification of usefull calligraphies
\newcommand{\mbb}[1]{\mathbb{#1}}			%
\newcommand{\mbf}[1]{\mathbf{#1}}			%
\newcommand{\msf}[1]{\mathsf{#1}}		       %
\newcommand{\mait}[1]{\mathit{#1}}			%
\newcommand{\mfrk}[1]{\mathfrak{#1}}		%
\newcommand{\tbf}[1]{\textbf{#1}}				%
\newcommand{\tsf}[1]{\textsf{#1}}				%
\newcommand{\tit}[1]{\textit{#1}}					%
\newcommand{\trm}[1]{\textrm{#1}}
\newcommand{\tref}{\text{ref}}
\newcommand{\tkin}{\text{kin}}				%
\newcommand{\noi}{\noindent}
\newcommand{\Frac}{\displaystyle\frac}
\newcommand{\Int}{\displaystyle\int}
\newcommand{\Sum}{\displaystyle\sum}
\newcommand{\Bigcup}{\displaystyle\bigcup}
\newcommand{\Max}{\displaystyle\max}
\newcommand{\Min}{\displaystyle\min}
\newcommand{\Eq}[1]{equation {(\ref{#1})}}
\newcommand{\Q}{\mathbf{Q}}
\renewcommand{\S}{\mathbf{S}}
\renewcommand{\u}{\mathbf{u}}
\newcommand{\w}{\mathbf{w}}
\newcommand{\m}{\mathbf{m}}
\newcommand{\F}{\mathbf{F}}
\newcommand{\f}{\mathbf{f}}
\newcommand{\h}{\mathbf{h}}
\newcommand{\etaB}{\eta_{\mathsmaller{B}}}
\newcommand{\etaPL}{\eta_{\mathsmaller{PL}}}
\newcommand{\etaHB}{\eta_{\mathsmaller{HB}}}
\newcommand{\tauHB}{\tau_{\mathsmaller{HB}}}
\newcommand{\cs}[1]{c_{\textrm{s},#1}}	% shear velocity in fluid
\newcommand{\cb}[1]{c_{\textrm{b},#1}}	% bulk velocity in fluid
\newcommand{\cp}[1]{c_{\textrm{p},#1}}	% compression or p-wave velocity in pure phase

\newcommand{\Cfast}{C_\textrm{fast}}	% mixture fast characteristic velocity
\newcommand{\Cslow}{C_\textrm{slow}}	% mixture slow characteristic velocity
\newcommand{\Cshear}{C_\textrm{shear}}	% mixture shear characteristic velocity
\newcommand{\csh}{c_{\textrm sh}}
\newcommand{\Vfast}{V_\textrm{fast}}	% mixture fast p-wave V(omega)
\newcommand{\Vslow}{V_\textrm{slow}}	% mixture slow p-wave V(omega)
\newcommand{\Vshear}{V_\textrm{shear}}	% mixture shear sound wave
\newcommand{\Smix}{S}

\newcommand{\uphi}{{v_\theta}} % angular velocity
\newcommand{\wphi}{{w_\theta}}
\newcommand{\ur}{{v_r}} % radial velocity
\renewcommand{\wr}{{w_r}}
%\renewcommand{\v}{\mathbf{v}}
\newcommand{\B}{\mathbf{B}}
\newcommand{\Ell}{\mathcal{L}}
\newcommand{\emm}{m}
\newcommand{\dxx}[1]{ \partial_{xx} #1 }
\newcommand{\dyy}[1]{ \partial_{yy} #1 }
\newcommand{\dzz}[1]{ \partial_{zz} #1 }
\newcommand{\HRule}[1]{ {\centering \rule{#1\linewidth}{0.1mm}} }
\newcommand{\DMPutwo}{$[\text{DMP}\!\to\!u2]$}
\newcommand{\PADDMPutwo}{$[\text{PAD}\!\to\!\text{DMP}\!\to\!u2]$}
\newcommand{\red}[1]{{\color{red} #1}}
\newcommand{\blue}[1]{{\color{blue} #1}}
\newcommand{\mygreen}{\textcolor[rgb]{0.0,0.60,0.}}
\newcommand{\myorange}{\textcolor[rgb]{0.6,0.0,0.}}
\newcommand{\oz}[1]{ \textcolor{red}   {\texttt{\textbf{OZ: #1}}} }
\newcommand{\rmd}{{\textrm d}}
\newcommand{\RS}{\text{RS}}



\newcommand{\bdm}{\begin{displaymath}}
    \newcommand{\edm}{\end{displaymath}}

\newcommand{\bea}{\begin{eqnarray} }
    \newcommand{\eea}{\end{eqnarray} }

\newcommand{\apriori}{\textit{a priori} }
\newcommand{\aposteriori}{\textit{a posteriori} }

\newcommand{\dev}{\textnormal{dev}}

\renewcommand{\div}{{\nabla \cdot}}
\newcommand{\divh}{{\nabla_h \cdot}}

\renewcommand{\AA}{{\tensor{A}}}
\newcommand{\aaa}{{\boldsymbol{a}}}
\newcommand{\DD}{{\mathbf{D}}}
\newcommand{\HH}{{\mathbf{H}}}
\newcommand{\sfA}{{\mathsf{A}}}
\newcommand{\sfB}{{\mathsf{B}}}
\newcommand{\GG}{{\tensor{G}}}
\newcommand{\g}{{\tensor{g}}}
\newcommand{\rr}{{\mathbf{r}}}
\newcommand{\ee}{{\mathbf{e}}}
\newcommand{\bb}{{\mathbf{b}}}
\newcommand{\hh}{{\mathbf{h}}}
\newcommand{\dd}{{\mathbf{d}}}
\newcommand{\vv}{{\mathbf{v}}}
\newcommand{\uu}{{\mathbf{u}}}
\newcommand{\mcE}{{\mathcal{E}}}
\newcommand{\calI}{\mathcal{I}}
\newcommand{\EE}{{\mathbf{E}}}
\newcommand{\BB}{{\mathbf{B}}}

\newcommand{\FF}{{\boldsymbol{F}}}
\newcommand{\II}{{\tensor{I}}}
\newcommand{\JJ}{{\mathbf{J}}}
\newcommand{\QQ}{{\mathbf{Q}}}
\renewcommand{\SS}{{\mathbf{S}}}
\newcommand{\QV}{{\mathbf{V}}}
\newcommand{\PP}{{\mathbf{P}}}
%\newcommand{\SS}{{\boldsymbol{S}}}
\newcommand{\WW}{{\mathbf{W}}}
\newcommand{\ww}{{\bm{w}}}
\newcommand{\wbf}{{\mathbf{w}}}
\newcommand{\pp}{{\mathbf{p}}}
\newcommand{\qq}{{\mathbf{q}}}
\newcommand{\nn}{{\bm{n}}}

\newcommand{\Id}{{\boldsymbol{I}}}
\newcommand{\tr}{\textnormal{tr}}
\newcommand{\BS}{{\tensor{\sigma}}}
\renewcommand{\Re}{\textnormal{Re}}
\newcommand{\transpose}{{\textrm { T}}}
\newcommand{\Bi}{\textrm{Bi}}
\newcommand{\mf}{c}			% mass fraction, which is usually c_1
\newcommand{\vf}{\alpha}	% volume fraction, which is usually \alpha_1
\newcommand{\indd}[1]{_{#1}}
\newcommand{\indu}[1]{^{#1}}
\newcommand{\rrat}[1]{\frac{\rho\indd{#1,\tref}}{\rho\indd{\tref}}}
\newcommand{\dhcentral}{\partial_{h}^{s,c}}
\newcommand{\dhp}{\partial_{h}^{s,+}}
\newcommand{\dhm}{\partial_{h}^{s,-}}
\newcommand{\dhs}{\partial_{h}^{s}}
\newcommand{\refn}{\text{ref}}

\newcommand*\samethanks[1][\value{footnote}]{\footnotemark[#1]}

\newcommand{\ip}[1]{\textcolor{magenta}{[IP: #1]}}
\newcommand{\at}[1]{\textcolor{blue}{[AT: #1]}}
\newcommand{\er}[1]{\textcolor{brown}{[ER: #1]}}

\newtheorem{theorem}{Theorem}[section]
\newtheorem{lemma}[theorem]{Lemma}
\newtheorem{corollary}[theorem]{Corollary}
\newtheorem{remark}[theorem]{Remark}
\newtheorem{assumption}[theorem]{Assumption}
\newtheorem{definition}[theorem]{Definition}


%%%%%%%%%%%%%%%%%%%%%%%%%%%%%%%%%%%%%%%%%%%%%%%%%%%%%%%%%%%%%%%%%%%%%%%%%%%%%%%%%%%%%%%%%%%%%%%%
% genrate filenames list of figures:
\usepackage{letltxmacro}


\LetLtxMacro\davidsincludegraphics\includegraphics

\makeatletter

%\RenewDocumentCommand{\includegraphics}{sO{}mo}{%
    %	\IfBooleanTF{#1}{%
        %		\davidsincludegraphics*[#2]{#3}%
        %	}{%
        %		\davidsincludegraphics[#2]{#3}%
        %	}%
    %	\begingroup
    %	% Trying to determine the extension
    %	\def\loc@l@ext{}
    %	\IfValueTF{#4}{%
        %		\def\loc@l@ext{#4}%
        %	}{%
        %		\IfFileExists{#3}{%
            %		}{%
            %			\IfFileExists{#3.pdf}{%
                %				\edef\loc@l@ext{.pdf}%
                %			}{%
                %				\IfFileExists{#3.jpg}{%
                    %					\edef\loc@l@ext{.jpg}%
                    %				}{%
                    %					\edef\loc@l@ext{.png}%
                    %				}%
                %			}%
            %		}%
        %	}%

    %	\advance\c@figure by \@ne
    %	\addtocontents{lfn}{\thefigure\space #3\loc@l@ext}
    %	\endgroup
    %}

\def\@starttocbutdonotshowit#1{%
    \begingroup
    \makeatletter
    \if@filesw
    \expandafter\newwrite\csname tf@#1\endcsname
    \immediate\openout \csname tf@#1\endcsname \jobname.#1\relax
    \fi
    \@nobreakfalse
    \endgroup}


\newcommand{\listoffigurenumbernames}{%
    \@starttocbutdonotshowit{lfn}%
}

%%%%%%%%%%%%%%%%%%%%%%%%%%%%%%%%%%%%%%%%%%%%%%%%%%%%%%%%%%%%%%%%%%%%%%%%%%%%%%%%%%%%%%%%%%%%%%%%
% Allow the use of \tensor{.} for both Latin and Greek letters
%%%%%%%%%%%%%%%%%%%%%%%%%%%%%%%%%%%%%%%%%%%%%%%%%%%%%%%%%%%%%%%%%%%%%%%%%%%%%%%%%%%%%%%%%%%%%%%%

\DeclareMathAlphabet{\mathsfbi}{OT1}{\sfdefault}{bx}{sl}
\DeclareMathVersion{sfletters}
\SetSymbolFont{letters}{sfletters}{OML}{ntxsfmi}{b}{it}

\makeatletter
\newcommand{\mathbfsbilow}[1]{%
    \text{\mathversion{sfletters}$\m@th#1$}%
}
\DeclareRobustCommand{\tensor}[1]{%
    \begingroup
    \ifcat\noexpand #1\relax
    % assume Greek letter
    \edef\greek@test{\detokenize{#1}}%
    \edef\greek@test{\expandafter\@cdr\greek@test\@nil}%
    \edef\greek@test{\expandafter\@car\greek@test\@nil}%
    \edef\x{\the\lccode\expandafter`\greek@test}%
    \edef\y{\number\expandafter`\greek@test}%
    \ifnum\x=\y\relax
    % the command name starts with a lower-case letter
    \mathbfsbilow{#1}%
    \else
    \mathsfbi{#1}%
    \fi
    \else
    \mathsfbi{#1}%
    \fi
    \endgroup
}
\makeatother

%%%%%%%%%%%%%%%%%%%%%%%%%%%%%%%%%%%%%%%%%%%%%%%%%%%%%

% DOC BEGINNING

\newfont{\numerikEleven}{ecrm1000}
\newfont{\numerikTen}{cmss10}
\newfont{\numerikNine}{cmss9}
\newfont{\numerikEight}{cmss8}

% New commands for the revision :
%\newcommand{\revOne}[2]{\textcolor{Red}{ [Rev1, Q.#1]: }\textcolor{Red}{#2}}
%\newcommand{\revTwo}[2]{\textcolor{Green}{ [Rev2, Q.#1]: }\textcolor{Green}{#2}}
%\newcommand{\revThree}[2]{\textcolor{Blue}{ [Rev3, Q.#1]: }\textcolor{Blue}{#2}}

\newcommand{\revOne}[1]{\textcolor{Red}{#1}}
\newcommand{\revTwo}[1]{\textcolor{Green}{#1}}
\newcommand{\revThree}[1]{\textcolor{Blue}{#1}}


%=========================================================================

%\listfiles
\begin{document}

    % TITLE
    \title{
        \textbf{Solving the implicit system of the SATh-upwind scheme}
    }
    %-------------------------------------------------------
    %-------------------------------------------------------
    % AUTHORS for revtex4-1 documentclass
    %\author{Ilya Peshkov}
    %\email{peshkov@math.nsc.ru}
    %\affiliation{Department of Civil, Environmental and Mechanical Engineering,
        %	University of Trento, Via Mesiano 77, 38123 Trento, Italy}
    %
    %\author{Michael Dumbser}
    %\email{michael.dumbser@unitn.it}
    %\affiliation{Department of Civil, Environmental and Mechanical Engineering,
        %	University of Trento, Via Mesiano 77, 38123 Trento, Italy}
    %
    %\author{Evgeniy Romenski}
    %\email{evrom@math.nsc.ru}
    %\affiliation{Sobolev Institute of Mathematics, 4 Acad. Koptyug Avenue and
        %Novosibirsk State University, 2 Pirogova Str., Novosibirsk, Russia}

    % AUTHORS for article documentclass
    \author{
        Andrea Thomann\thanksgap{-0.5ex}
        %        \samethanks[1]
        %        \textsuperscript{,}
        %        \thanksgap{-1.25ex}
        \thanks{Universit\'e de Strasbourg, CNRS, Inria, IRMA, F-67000 Strasbourg, France,
            (\href{mailto:andrea.thomann@inria.fr}{andrea.thomann@inria.fr})}
        %        \textsuperscript{,$\ast$}
    }
    \thanksmarkseries{arabic}

    \maketitle %\maketitle must follow title, authors, abstract and \pacs

    \date{\today}
    %    {\let\thefootnote\relax\footnote{\hspace{0.25cm}$^* $Corresponding author}}

    \paragraph{Abstract:}
    \paragraph{Keywords:}


    %-----------------------------------
    % CONTENTS
    %  This will dessapear in the submitted version
    %\tableofcontents

    %\begin{multicols}{2}
    \section{The SATh-upwind scheme for linear transport}
    Following the slides of the talk of Arbogast page 13 with a linear flux
    \begin{equation}
        f(u) = a u,
    \end{equation}
    The scheme reads in a finite volume context with $\lambda_i = a\Delta t/\Delta x_i$
    \begin{align}
\begin{split}
            u_i^{n+1} - u_i^n &= - \lambda_i \left((1-\theta_i) u_i^n + \theta_i u_i^{n+1} - (1-\theta_{i-1})u_{i-1}^n - \theta_{i-1} u_{i-1}^{n+1}\right) \\
        \tilde u_i^{n+1} - u_i^n &= - \frac{\lambda_i}{2} \left((1-\theta_i^2) u_i^n + \theta_i^2 u_i^{n+1} - (1-\theta_{i-1}^2)u_{i-1}^n - \theta_{i-1}^2 u_{i-1}^{n+1}\right).
\end{split}
    \end{align}
    Let $w_i := u_i^{n+1} - u_i^n$ and $v_i := \tilde{u}_i^{n+1}-u_i^n$, then after some rearrangement of the terms, we can write the scheme as follows
    \begin{align}
        \label{eq.Nonlin_system}
        \begin{split}
            w_i + \lambda_i \theta(v_i/w_i) w_i &= - \lambda_i \left(u_i^n - \theta(v_{i-1}/w_{i-1})w_{i-1} - u_{i-1}^{n+1}\right)\\
        v_i + \frac{\lambda_i}{2} \theta(v_i/w_i)^2 w_i &= - \frac{\lambda_i}{2} \left(u_i^n - \theta(v_{i-1}/w_{i-1})^2w_{i-1} - u_{i-1}^{n+1}\right),
        \end{split}
    \end{align}
    where we used
    \begin{equation}
        \label{eq.Theta}
        \theta\left(\frac{v_i}{w_i}\right):= \begin{cases}
            \max\left(\theta_{\min},\frac{v_i}{w_i}\right) & \text{ if }\quad |w_i| > 0\\
            \theta^\ast & \text{else.}
        \end{cases}
    \end{equation}
    Note that the right hand side of \eqref{eq.Nonlin_system} is known if the system is solved in upwind direction, i.e. starting on cell 1 with known boundary data, then solving on cell 2, where data on cell 1 is known etc.
    \subsection{Newton}
    It remains to solve the non-linear system on cell $i$. We use Newton, i.e. we solve the system
    \begin{equation}
        F(\xx) = 0, \quad , F: \mathbb{R}^2 \to \mathbb{R}^2, \quad F(\xx) = x + \lambda_i
        \begin{pmatrix}
            \theta\left(\frac{x_1}{x_2}\right) \\\frac{1}{2}\theta\left(\frac{x_1}{x_2}\right)^2
        \end{pmatrix} x_1 - \text{rhs}
    \end{equation}
    where $\xx = (w_i, v_i)^T$ and $\text{rhs}$ denotes the known right hand side values of \eqref{eq.Nonlin_system}.

    For the Newton algorithm, we need the Jacobian of $F$.
    It is given by three cases
    \begin{enumerate}
        \item If $|w_i| > \varepsilon$ and $\frac{v_i}{w_i} > \theta_{\min}$ then
        \begin{equation}
            \nabla_\xx F = J = \begin{pmatrix}
                1 & \lambda_i \\
                - \frac{v_i^2 \lambda_1}{2 w_i^2} & 1 + \frac{\lambda_i}{w_i}
            \end{pmatrix}
        \end{equation}
        \item Else if $|w_i| > \varepsilon$ and $\frac{v_i}{w_i} \leq \theta_{\min}$ then
                \begin{equation}
            \nabla_\xx F = J = \begin{pmatrix}
                1 + \theta_{\min} \lambda_i & 0 \\
                \frac{\theta_{\min}^2}{2}\lambda_i & 1
            \end{pmatrix}
        \end{equation}
        \item Else
         \begin{equation}
             \nabla_\xx F = J = \begin{pmatrix}
                 1 + \theta^\ast \lambda_i & 0 \\
                 \frac{(\theta^\ast)^2}{2}\lambda_i & 1
             \end{pmatrix}
         \end{equation}
    \end{enumerate}
    \subsection{Fixpoint iteration}
    We propose the following fixpoint iteration to solve \eqref{eq.Nonlin_system}.
    The idea is, to take $\theta$ in \eqref{eq.Theta} at the previous iteration.

    Let $\theta^{(0)} = \theta^\ast$, then in the $k$-th iteration to obtain $w_i$, we set
    \begin{align}
        \label{eq.Nonlin_system_FP}
        \begin{split}
            w_i^{(k)} + \lambda_i \theta^{(k-1)} w_i^{(k)} &= - \lambda_i \left(u_i^n - \theta(v_{i-1}/w_{i-1})w_{i-1} - u_{i-1}^{n+1}\right)\\
            v_i^{(k)} + \frac{\lambda_i}{2} (\theta^{(k-1)})^2 w_i^{(k)} &= - \frac{\lambda_i}{2} \left(u_i^n - \theta(v_{i-1}/w_{i-1})^2w_{i-1} - u_{i-1}^{n+1}\right).
        \end{split}
    \end{align}
    Note that again, the right hand side of \eqref{eq.Nonlin_system_FP} is known, since we have already solved the equations on cell $i-1$.
    Once we have obtained $w_i^{(k)}$ from the now linear system \eqref{eq.Nonlin_system_FP}, we set the new $\theta$ as
    \begin{equation}
        \theta^{(k)} = \theta\left(\frac{v_i^{(k)}}{w_i^{(k)}}\right)
    \end{equation}
    and solve again until a suitable error is achieved, for instance $| w_i^{(k)} - w_i^{(k-1)}| < \hat{\varepsilon}$, i.e. the relative error is changing only marginally from iteration to iteration, which means that we have converged to a solution.
    Note, that in \eqref{eq.Nonlin_system_FP} only the first equation has to be solved implicitly, and then $v_i^{(k)}$ can be updated explicitly.
    However, the implicit system is linear and given by
    \begin{equation}
         w_i^{(k)} = \left(1 + \lambda_i \theta^{(k-1)}\right)^{-1}\text{rhs}_1.
    \end{equation}
    It is well-posed, since $\lambda_i$ and $\theta$ are always positive.




    %    %%=============================================================================
    %    %%==========  B I B L I O G R A P H Y
    %    \bibliographystyle{plain}
    %    \bibliography{biblio.bib}
    %    %%=============================================================================
    %    %\printbibliography
    %    %
    %    %\listoffigurenumbernames

\end{document}

